\documentclass[12pt]{letter}
\usepackage[margin=1in]{geometry}
\usepackage{url}
\usepackage[colorlinks]{hyperref}
\usepackage{graphicx}

\name{William Stafford Noble, Professor\\
Department of Genome Sciences\\
University of Washington}

\thispagestyle{empty}

\begin{document}

\begin{letter}{Dr.\ Guy Jones, Senior Editor \\
{\em Scientific Data}}

\opening{Dear Dr.\ Jones,}

I am pleased to submit a manuscript, ``A multi-species benchmark for training and validating large scale mass spectrometry proteomics machine learning models,'' for consideration for publication in {\em Scientific Data}.

This manuscript describes a dataset that was curated for development of our Casanovo {\em de novo} peptide sequencing model.
The model is described in the paper, ``Sequence-to-sequence translation from mass spectra to peptides with a transformer model,'' which is currently in press in {\em Nature Communications}.
The dataset consists of annotated peptide mass spectra derived from nine different species.
This nine-species benchmark is modeled after the dataset used in a seminal paper from 2017, ``{\em De novo} peptide sequencing by deep learning,'' published by Tran {\em et al.}\ in {\em PNAS}.
In revisiting that benchmark, we identified a series of significant problems with the original processing pipeline, including incorrect search settings and incomplete separation of train and test sets.
These problems are described in more detail in the accompanying manuscript.
Our revised benchmark fixes these problems and makes the data, as well as intermediate files, available via Zenodo.

Please note that, in addition to this benchmark dataset, our forthcoming Casanovo paper makes use of a separate dataset containing 30 million human spectra, derived from the MassIVE-KB database.
This dataset was initially developed for our paper, ``A learned embedding for efficient joint analysis of millions of mass spectra,'' by Bittremieux {\em et al.}, published in {\em Nature Methods} in 2022.
We are preparing separate manuscript describing this dataset, which we intend to submit to {\em Scientific Data} in the coming weeks.

Thank you for considering our manuscript for review.

\closing{Best regards,}
\end{letter}

\end{document}
